\documentclass[../../main]{subfiles}

\begin{document}

\section{Simulation}
In this section, you'll be bombarded with a bunch of tasks designed to get you acquainted with the tellurium
PyCoTools. Since both tools work with Antimony strings, you can use both tools seemlessly
with your model.

\begin{Task}[label=TimeSeries]{Run A time series}
Run a time series with both tellurium and PyCoTools on your model. Plot the time series using both tools.
    Plot the data using `matplotlib'.
\end{Task}

\begin{Task}[label=SteadyState]{Run A steady state simulation}
    Run a steady state simulation with tellurium. Note that the steady state task is not supported by PyCoTools, so you'll have to open the model with the user interface
    if you want to run a steady state with COPASI.
\end{Task}

\begin{Task}[label=CreateFakeData]{Create some fake data for fitting to your model}
    To demonstrate parameter estimation, the quickest way is to simulate some data from the
    model you have built and configure a parameter estimation using this simulated data.
    \begin{itemize}
        \item Create a new variable in your `\_\_init\_\_.py' to hold the full path to a
        (yet non-existent) data file under the `COPASI\_FORMATTED\_DATA\_DIR'.
        As this is a global variable (not to be confused with a `global quantity' inside the model)
        used elsewhere in the project, make the name all capital letters.
        \item Run a time series using either tellurium or pycotools.
        \item Ensure your variable names are exactly the same as those used in your model. If you have used
        tellurium, you should convert your results object to a
        `\href{https://pandas.pydata.org/pandas-docs/stable/reference/api/pandas.DataFrame.html}{pandas.DataFrame} so you can easily
        modify the names of the columns (accessed with `df.columns') and easily save the data to file.
        \item Save the data to the newly created global variable (using `pandas.csv').
    \end{itemize}
\end{Task}

\begin{Task}[label=ManualParameterEstimation]{Parameter Estimation in Copasi}
    Open your model in copasi and set up a parameter estimation with your simulated data. Pick the
    parameters you want to estimate, define some plots (bottom right corner) and try out a few algorithms.
    Make sure the `random initial conditions' button is checked. Play around with the algorithms hyperparameters and
    notice how they affect convergance.
\end{Task}

\begin{Task}[label=AutoParameterEstimation]{Parameter Estimation Using PyCoTools}
    Configure and run a parameter estimation using PyCoTools. PyCoTools essentially automates the procedure
    from the previous task.
    Play around with the options in PyCoTools by configuring the a parameter estimation to setup all the following
    parameter estimations. When changing from one configuration to another, change the `problem' or `fit' settings
    to isolate new results from previous runs.

    \begin{itemize}
        \item Use the `problem' and `fit' settings to organise your
        \item Estimate all global quantities
        \item Estimate all metabolites
        \item Estimate all global quantities and metabolites
        \item Change the algorithm you are using to something different. Also change the hyperparameters (like iteration limit or population size)
        \item Run a parameter estimation in `parallel' mode
        \item Use the `prefix' option to be selective about parameters you are estimating.
        \item \note{PyCharms find and replace feature is particularly useful when renaming parameters to include the prefix}
    \end{itemize}
\end{Task}

\begin{Task}[label=VizPE]{Visualising Parameter Estimation Results}
    Now that we have some parameter estimation data we can produce some diagnostic plots
    to evaluate the estimations performance and compare simulated versus experimental data.

    \begin{itemize}
        \item Plot a waterfall plot
        \item Plot a boxplot of you parameter distributions
        \item Plot a violin of you parameter distributions
        \item Plot the solution of the best fitting parameters against the experimental data. Note
        that there are features in the `pycotools.viz' module for this.
        \item Read the parameter estimation data into Python using the `viz.Parse' class.
        \item Explore the parameter estimation data using pandas.
        \item Insert best parameters into the model.
        \item Insert best parameters into the model and open with the COPASI UI. Simulate the
        parameter estimations using the current solution statics method to visualise the experiment
        versus simulated data within copasi.
        \item Insert the second best parameter set into the model and repeat the above.
    \end{itemize}

\end{Task}

\begin{Task}[label=ProfileLikelihoods]{Profile Likelihoods}
    Perform an identifiability analysis using the Profile likelihood method using PyCoTools. Due to a
    unfortunate mishap, I forgot to rewrite the code for visualising profile likelihoods after jazzing up
    PyCoTools. Optionally plot the profile likelihoods yourself using matplotlib and seaborn. You can parse the
    data into Python using the usual `viz.Parse' 'class.
\end{Task}

\begin{Task}[label=AdvancedInterface]{Using the low level interface}
    The ParameterEstimation.Context takes care of all of the detailed arguments required
    to configure a parameter estimation. However, sometimes it is necessary to have a more
    fine grained level of control over the configuration.

    The ParameterEstimation.Context class exists in order to construct a ParameterEstimation.Config object.
    However it is also possible to just create the Config object yourself.
    \begin{itemize}
        \item Start by using the Context class to create a template for your configuration.
        \item When printed, the config object in yaml format.
        \item Print the resulting config object and copy and paste it into a new file with a .yaml extension.
        \item
    \end{itemize}

\end{Task}




\end{document}






