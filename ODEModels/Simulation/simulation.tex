\documentclass[../../main]{subfiles}

\begin{document}

\section{Simulation}
In this section, you'll be bombarded with a bunch of tasks designed to get you acquainted with the tellurium
PyCoTools. Since both tools work with Antimony strings, you can use both tools seemlessly
with your model.

You should use the tellurium and PyCoTools documentation to work out how to do the

\begin{Task}[label=TimeSeries]{Run A time series}
Run a time series with both tellurium and PyCoTools on your model. Plot the time series using both tools.
    Plot the data using `matplotlib'.
\end{Task}

\begin{Task}[label=CreateFakeData]{Create some fake data for fitting to your model}
    To demonstrate parameter estimation, the quickest way is to simulate some data from the
    model you have built and configure a parameter estimation using this simulated data.
    \begin{itemize}
        \item Create a new variable in your `\_\_init\_\_.py' to hold the full path to a
        (yet non-existent) data file under the `COPASI\_FORMATTED\_DATA\_DIR'.
        As this is a global variable (not to be confused with a `global quantity' inside the model)
        used elsewhere in the project, make the name all capital letters.
        \item Run a time series using either tellurium or pycotools.
        \item Ensure your variable names are exactly the same as those used in your model. If you have used
        tellurium, you should convert your results object to a
        `\href{https://pandas.pydata.org/pandas-docs/stable/reference/api/pandas.DataFrame.html}{pandas.DataFrame} so you can easily
        modify the names of the columns (accessed with `df.columns') and easily save the data to file.
        \item Save the data to the newly created global variable (using `pandas.csv').
    \end{itemize}
\end{Task}

\begin{Task}[label=ManualParameterEstimation]{Parameter Estimation in Copasi}
    Open your model in copasi and set up a parameter estimation with your simulated data. Pick the
    parameters you want to estimate, define some plots (bottom right corner) and try out a few algorithms.
    Make sure the `random initial conditions' button is checked. Play around with the algorithms hyperparameters and
    notice how they affect convergance.
\end{Task}

\begin{Task}[label=AutoParameterEstimation]{Parameter Estimation Using PyCoTools}
    Configure and run a parameter estimation using PyCoTools. PyCoTools essentially automates the procedure
    from the previous task.
    Play around with the options in PyCoTools by configuring the a parameter estimation to setup all the following
    parameter estimations. When changing from one configuration to another, change the `problem' or `fit' settings
    to isolate new results from previous runs.

    \begin{itemize}
        \item Use the `problem' and `fit' settings to organise your
        \item Estimate all global quantities
        \item Estimate all metabolites
        \item Estimate all global quantities and metabolites
        \item Change the algorithm you are using to something different. Also change the hyperparameters (like iteration limit or population size)
        \item Run a parameter estimation in `parallel' mode
        \item Use the `prefix' option to be selective about parameters you are estimating.
        \item \note{PyCharms find and replace feature is particularly useful when renaming parameters to include the prefix}
    \end{itemize}
\end{Task}

\begin{Task}[label=VizPE]{Visualising Parameter Estimation Results}
    That that we have

\end{Task}
    Plot time series
    plot parameter estimations
    run steady state with roadrunner and with copasi
    Insert parameters intoa model
    plotting the best fit.
    accessing parameter estimation data.
    using pycotools on the cluster
    using the advanced interface
    profile likelihoods





\end{document}






