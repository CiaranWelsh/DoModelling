\documentclass[../../main]{subfiles}

\begin{document}

\section{Simulation}

\begin{Task}[label=TimeSeries]{Run A time series}
Run a time series with both tellurium and PyCoTools on your model.
\end{Task}

\begin{Task}[label=CreateFakeData]{Create some fake data for fitting to your model}
    \begin{itemize}
        \item Create a new global variable representing the full path to a (yet non-existent) data file under the `COPASI\_FORMATTED\_DATA\_DIR'.
        \item Run a time series using either tellurium or pycotools.
        \item Ensure your variable names are exactly the same as those used in your model
        \item Save the data to the newly created global variable (using `pandas.csv').
    \end{itemize}
\end{Task}

\begin{Task}[label=ManualParameterEstimation]{Parameter Estimation in Copasi}
    Open your model in copasi and set up a parameter estimation with your simulated data. Pick the
    parameters you want to estimate and run a few algorithms, starting from random initial conditions.
\end{Task}

\begin{Task}[label=AutoParameterEstimation]{Parameter Estimation Using PyCoTools}
    Your parameter estimation shouldn't take long so take a moment to
    play around with the options available in PyCoTools. For the most part, they are the same as those available in COPASI.
    Configure the following estimations
    \begin{itemize}
        \item Estimate all global quantities
        \item Estimate all metabolites
        \item Estimate all global quantities and metabolites
        \item Change the algorithm you are using to something different. Also change the hyperparameters (like iteration limit or population size)
    \end{itemize}
\end{Task}




\end{document}






