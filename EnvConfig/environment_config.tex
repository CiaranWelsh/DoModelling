\begin{document}

    \section{Environment Configuration}

    \subsection{Python}
    Python is a program written in C. It is an executable file (i.e. a binary) that must be compiled and linked
    from source before you can use it. There are many `distributions' of Python, which essentially just means
    different people have compiled it and packaged it in slightly different ways.

    My favourite distribution of Python is \href{https://docs.conda.io/en/latest/miniconda.html}{Miniconda}. Miniconda
    and Anaconda are essentially the same, but Anaconda comes with a whole bunch of additional Python packages. These
    can be useful, but it is quicker to just use Miniconda.



    Anaconda and Miniconda (or just Conda) allows you to create isolated Python environments
    and switch between them easily. Think of each conda environment you have as a separate box that is
    kept separate from the other Python `boxes'. While the full documentation can be found \href{https://docs.conda.io/projects/conda/en/latest/user-guide/tasks/manage-environments.html}{here}
    the commands to create a conda environment


%    \begin{minted}{}
%        x = 4
%    \end{minted}




\end{document}