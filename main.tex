\documentclass{article}
\usepackage[a4paper, total={6in, 8in}]{geometry}
\usepackage[utf8]{inputenc}
\usepackage{subfiles}
\usepackage{hyperref}
\hypersetup{
     colorlinks   = true,
     citecolor    = blue
}


% for coloured boxes
\usepackage[most]{tcolorbox}
%textmarker style from colorbox doc
\tcbset{textmarker/.style={%
enhanced,
parbox=false,boxrule=0mm,boxsep=0mm,arc=0mm,
outer arc=0mm,left=6mm,right=3mm,top=7pt,bottom=7pt,
toptitle=1mm,bottomtitle=1mm,oversize}}

% define new colorboxes
\newtcolorbox{hintBox}{textmarker,
borderline west={6pt}{0pt}{yellow},
colback=yellow!10!white}

\newtcolorbox{importantBox}{textmarker,
borderline west={6pt}{0pt}{red},
colback=red!10!white}

\newtcolorbox{noteBox}{textmarker,
borderline west={6pt}{0pt}{green},
colback=green!10!white}

\newtcolorbox{rememberBox}{textmarker,
borderline west={6pt}{0pt}{blue},
colback=blue!10!white}

\newtcolorbox[auto counter, number within=chapter,
number freestyle={\noexpand\thesection.\noexpand\arabic{\tcbcounter}}
]{Task}[2][]{%
enhanced,
breakable,
fonttitle=\bfseries,
title=Task~\thetcbcounter: #2,
#1
}

\usepackage{subfig}


% define commands for easy access
\newcommand{\note}[1]{\begin{noteBox}
                          \textbf{Note:} #1
\end{noteBox}}
\newcommand{\warning}[1]{\begin{hintBox}
                             \textbf{Warning:} #1
\end{hintBox}}
\newcommand{\important}[1]{\begin{importantBox}
                               \textbf{Important:} #1
\end{importantBox}}

\newcommand{\task}[1]{\begin{taskBox}
                          \textbf{Task:} #1
\end{taskBox}}


\newcommand{\remember}[1]{\begin{rememberBox}
                          \textbf{Remember:} #1
\end{rememberBox}}

\usepackage{minted}


\usepackage{graphicx}
\usepackage{parskip}
%always make last package imported
%\usepackage{cleveref}
\usepackage{xparse}

\NewDocumentCommand{\code}{v}{%
\texttt{\textcolor{grey}{#1}}%
}

\newcommand{\bash}[1]{\mintinline{bash}{#1}}
\newcommand{\python}[1]{\mintinline{python}{#1}}

\usepackage{xcolor}
\usepackage{listings}
\usepackage{natbib}
\bibliographystyle{abbrvnat}

\title{Modelling stuff}
\author{Ciaran Welsh}

\begin{document}
    \maketitle
    \tableofcontents

    \section{Introduction}
    This document is essentially a list of practical computational exercises designed to try and distribute some of what
    I have learnt over the years. Importantly, what is described here is only one way of organising
    a modelling project, but it has been refined through experience and is what works best for me.
    Hopefully it will also work for you.

    \section{Reading material}
    Here are some suggested papers to read preferably before the tutorial sessions. The better you understand this
    stuff before the tutorials, the more you'll get out of them.
%
    \begin{itemize}
        \item \cite{raue2013lessons}
        \item \cite{alon2007network}
        \item \href{https://www.tutorialspoint.com/python/python_modules.htm}{Python Modules, packages and subpackages}
        \item \href{https://product.hubspot.com/blog/git-and-github-tutorial-for-beginners}{Introduction to GitHub}
    \end{itemize}

    \subfile{EnvConfig/environment_config.tex}
    \subfile{ODEModels/ode_models.tex}

    \bibliography{references}

\end{document}











