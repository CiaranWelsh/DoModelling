%%
%% Author: ncw135
%% 13/05/2019
%%

% Preamble
\documentclass[11pt]{article}

% Packages
\usepackage{amsmath}
\usepackage{graphicx}
\usepackage[a4paper]{geometry}
\usepackage{cleveref}
\usepackage{subcaption}
% Document

\title{Task2: Parameter estimation}
\begin{document}
\maketitle
    In task1 you built a model from a full description of the topology, initial conditions and parameters. This
    time you are given the topology and some experimental data, but you will have to find the parameters yourself. Its
    a good idea, when you get time, to read some literature on parameter estimation of ODE models. Its not essential,
    since mostly Copasi takes care of the details, but it is helpful to have at least a rudimentary understanding of
    what is going on...

    In model 2, A is activated by the presence of S with Hill kinetics. Hill kinetics are a common way of modelling
    receptor activation. Then we have a small phosphorylation cascade starting at Ap which activates B which in turn
    activates C. Both forwards and backwards reactions follow michaelis-menten kinetics except for the activation of B
    which is instead competitively inhibited by I. Note that the competitive inhibition rate law converges to the
    michaelis menten when the inhibitor is not present. 


\end{document}